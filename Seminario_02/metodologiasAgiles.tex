\documentclass{article}

% Language setting
\usepackage[spanish]{babel}

% Set page size and margins
\usepackage[letterpaper,top=2cm,bottom=2cm,left=3cm,right=3cm,marginparwidth=1.75cm]{geometry}

% Useful packages
\usepackage{amsmath}
\usepackage{graphicx}
\usepackage[colorlinks=true, allcolors=blue]{hyperref}
\usepackage{setspace}

\title{\vspace{-2cm}\large\textbf{Buenas prácticas para el desarrollo de aplicaciones ágiles.}}
\author{}
\date{\vspace{-2.2cm}}

\begin{document}

\maketitle

\begin{center}
    \textbf{P. J. Aguilar Vaides} \\
    7690-20-8927 Universidad Mariano Gálvez \\
    Seminario de Tecnologías de Información \\
    \href{mailto:jperez@gmail.com}{paguilarv@miumg.edu.gt}
\end{center}

\section{Resumen}

Las metodologías de desarrollo de software ágil han revolucionado la gestión de proyectos, enfocándose en la flexibilidad, la colaboración y la entrega continua de valor, a diferencia de los enfoques tradicionales más rígidos. La metodología ágil se basa en el "Manifiesto Ágil", que promueve iteraciones cortas, colaboración constante, entrega continua, flexibilidad y un enfoque en el cliente.

Las metodologías ágiles más usadas incluyen Scrum, Kanban, Extreme Programming (XP) y Lean Software Development. Scrum, por ejemplo, divide el proyecto en sprints y utiliza reuniones diarias para evaluar el progreso.

El desarrollo ágil implica la identificación de requisitos, planificación de sprints, desarrollo, revisión y retroalimentación, y retrospectivas para mejorar continuamente. Esto permite que el software se adapte rápidamente a los cambios y se enfoque en la satisfacción del usuario.\\

En cuanto a la metodología de la investigación, es crucial seguir un enfoque riguroso y sistemático para garantizar la validez y fiabilidad de los hallazgos. Los tipos de investigación incluyen la cuantitativa, que se centra en la recolección y análisis de datos numéricos; la cualitativa, que se enfoca en la comprensión de fenómenos sociales a través de datos no numéricos; y la mixta, que combina ambos enfoques. El diseño de investigación puede ser descriptivo, exploratorio, experimental o correlacional. La recolección de datos puede realizarse mediante encuestas, entrevistas, observación y análisis documental, siempre asegurando la rigurosidad y sistematicidad del proceso. El análisis de datos puede ser cuantitativo, utilizando técnicas estadísticas, o cualitativo, empleando métodos como el análisis de contenido y el análisis temático.\\

\section{Palabras Clave}
\subsection{Objetivo}
Explicar las metodologías de desarrollo de software ágil, destacando sus características, beneficios, diferencias con enfoques tradicionales, y su implementación en proyectos para mejorar la flexibilidad, colaboración, y entrega continua de valor, resultando en un software de mayor calidad y satisfacción del cliente.
\begin{itemize}
    \item Colaboración
    \item Entrega continua
    \item Manifiesto Ágil
    \item Scrum
    \item Kanban
    \item Feedback
    \item Adaptabilidad
\end{itemize}

\section{Desarrollo ágil de software: ¿qué es y cómo funciona?}
\onehalfspacing
\subsection{Qué es el desarrollo ágil de software: principales metodologías y principios}
El desarrollo ágil de software surgió como respuesta a las limitaciones del modelo en cascada, que era rígido y difícil de adaptar a los cambios rápidos del mercado. Este enfoque innovador permite a los desarrolladores adaptar las soluciones a sus necesidades, dividiendo los proyectos en partes más pequeñas y entregando productos funcionales en ciclos cortos para una mejor adaptación a las necesidades del cliente.

\subsection{¿Qué es el desarrollo ágil de software?}
El desarrollo ágil se basa en seis pasos fundamentales: planificación, estudio de requisitos, diseño, codificación, evaluación y documentación. A diferencia del modelo en cascada, el desarrollo ágil permite trabajar en diversos elementos en paralelo, validando y probando pequeños componentes del proyecto continuamente, lo que garantiza alta calidad y mejores resultados. El trabajo en equipo y los encuentros informales durante todo el proceso son clave.

\subsection{Principios del desarrollo ágil}
\begin{enumerate}
    \item \textbf{Individuos y procesos:} El capital humano es más importante que los procesos y herramientas.
    \item \textbf{Cliente:} La colaboración del cliente es crucial en todas las etapas del proceso. 
    \item \textbf{Adaptación:} Capacidad de adaptarse rápidamente a los cambios del mercado y los requisitos del cliente. El desarrollo ágil enfatiza la entrega anticipada y la simplificación del software y su desarrollo.
\end{enumerate}

\subsection{Principales metodologías de desarrollo ágil}
\begin{enumerate}
    \item \textbf{Scrum:} Marco de trabajo para el desarrollo, entrega y mantenimiento de programas complejos.
    \item \textbf{Extreme Programming (XP):} Entorno para el desarrollo de software de alta calidad.
    \item \textbf{Kanban:} Sistema de información que controla el desarrollo de procesos en cada etapa del proyecto.
    \item \textbf{Test-Driven Development (TDD):} Proceso de desarrollo basado en la repetición de ciclos de pruebas del software.
\end{enumerate}

\section{Ciclo de vida de desarrollo de software según la metodología ágil}
El ciclo de vida del desarrollo de software ágil incluye las siguientes etapas:
\begin{enumerate}
    \item \textbf{Determinación del alcance y la prioridad de los proyectos}
    \begin{itemize}
        \item Planificación y priorización de proyectos.
        \item Definición de la oportunidad comercial y evaluación de la viabilidad técnica y económica.
        \item Decisión sobre qué proyectos seguir según su prioridad y viabilidad.
    \end{itemize}
    \item \textbf{Diagrama de requisitos para el sprint inicial}
    \begin{itemize}
        \item Identificación de requisitos en colaboración con las partes interesadas.
        \item Uso de diagramas de flujo de usuario o UML de alto nivel.
        \item Selección de miembros del equipo y asignación de recursos.
        \item Creación de cronogramas o mapas de proceso con carriles para delinear responsabilidades.
    \end{itemize}
    \item \textbf{Construcción/iteración}
    \begin{itemize}
        \item Trabajo en la primera iteración del proyecto, con el objetivo de lanzar un producto funcional al final del sprint.
        \item Incorporación de funcionalidad básica mínima.
        \item Posibilidad de sprints adicionales para expandir el producto.
    \end{itemize}
    \item \textbf{Puesta en producción de la iteración}
    \begin{itemize}
        \item Prueba del sistema por el equipo de control de calidad (QA).
        \item Corrección de defectos.
        \item Finalización del sistema y documentación del usuario.
        \item Implementación de la iteración en producción.
    \end{itemize}
    \item \textbf{Producción y soporte continuo para la versión del software}
    \begin{itemize}
        \item Mantenimiento del sistema funcionando sin problemas.
        \item Soporte continuo y capacitación a los usuarios.
        \item Finalización de esta fase cuando el soporte ha terminado o se planifica el retiro de la versión.
    \end{itemize}
    \item \textbf{Fase de retiro}
    \begin{itemize}
        \item Eliminación de la versión del sistema de producción.
        \item Reemplazo por una nueva versión o eliminación por obsolescencia.
    \end{itemize}
\end{enumerate}

\subsection{Planificación de un sprint de desarrollo de software con la metodología ágil}
Dentro del SDLC ágil, el trabajo se divide en sprints, con los siguientes pasos básicos:
\begin{itemize}
    \item \textbf{Planificación:} Reunión para establecer componentes de la próxima ronda de trabajo, priorización de tareas por el director de producto.
    \item \textbf{Desarrollo:} Diseño y desarrollo del producto de acuerdo con las pautas aprobadas.
    \item \textbf{Prueba/QA:} Pruebas exhaustivas y documentación de resultados antes de la implementación.
    \item \textbf{Implementación:} Presentación del producto en funcionamiento a las partes interesadas y los clientes.
    \item \textbf{Evaluación:} Solicitud de comentarios al cliente y recopilación de información para incorporarla en el próximo sprint.
\end{itemize}
Reuniones diarias permiten al equipo mantenerse al día con el progreso, hablar sobre conflictos y avanzar en el proceso. La flexibilidad y apertura al cambio son esenciales en esta metodología.

\section{Observaciones y comentarios}
El desarrollo ágil de software se destaca por su flexibilidad y capacidad de adaptación, permitiendo a los equipos de desarrollo ajustarse rápidamente a los cambios de requisitos y prioridades del cliente, una característica crucial en el entorno dinámico de la tecnología moderna. La comunicación constante y la colaboración estrecha entre los desarrolladores y las partes interesadas aseguran que el producto final se alinee con las expectativas y necesidades reales del cliente, lo que resulta en una mayor satisfacción del usuario final.
\section{Conclusiones:}
\begin{itemize}
    \item El desarrollo ágil de software ha revolucionado la manera en que se crean y entregan soluciones tecnológicas, adaptándose a las necesidades cambiantes del mercado y de los clientes. Esta metodología enfatiza la colaboración, la flexibilidad y la entrega continua de valor, permitiendo a los equipos de desarrollo responder rápidamente a los desafíos y oportunidades. 
    \item El objetivo del ciclo de vida del desarrollo de software con la metodología ágil es crear e implementar software que funcione lo antes posible, adaptándose rápidamente a los cambios y necesidades.
\end{itemize}

\section{Referecias:}
\begin{itemize}
    \item adminstark. (n.d.). Métodos ágiles en el desarrollo de software. Stark Cloud. Recuperado de https://www.starkcloud.com/starkcloud-blog/metodos-agiles-en-el-desarrollo-de-software
    \item Axarnet. (n.d.). Desarrollo ágil de software: ¿qué es y cómo funciona?. Recuperado de https://axarnet.es/blog/desarrollo-agil#:~:text=El%20desarrollo%20%C3%A1gil%20de%20software%20se%20basa%20en%20seis%20pasos,entrega%20la%20totalidad%20del%20software.
    \item Lucidchart. (n.d.). Etapas del desarrollo de metodologías de software ágiles. Recuperado de https://www.lucidchart.com/blog/es/ciclo-de-vida-del-desarrollo-de-software-segun-la-metodologia-agil
\end{itemize}

\end{document}
