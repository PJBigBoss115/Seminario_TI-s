\documentclass{article}

% Language setting
\usepackage[spanish]{babel}

% Set page size and margins
\usepackage[letterpaper,top=2cm,bottom=2cm,left=3cm,right=3cm,marginparwidth=1.75cm]{geometry}

% Useful packages
\usepackage{amsmath}
\usepackage{graphicx}
\usepackage[colorlinks=true, allcolors=blue]{hyperref}
\usepackage{setspace}

\title{\vspace{-2cm}\large\textbf{Ciberseguridad, seguridad Informática y Seguridad de la Información.}}
\author{}
\date{\vspace{-2.2cm}}

\begin{document}

\maketitle

\begin{center}
    \textbf{P. J. Aguilar Vaides} \\
    7690-20-8927 Universidad Mariano Gálvez \\
    Seminario de Tecnologías de Información \\
    \href{mailto:jperez@gmail.com}{paguilarv@miumg.edu.gt}
\end{center}

\section{Resumen}

El estudio de la ciberseguridad, seguridad informática y seguridad de la información es fundamental en el entorno digital moderno para proteger sistemas, redes y datos frente a ataques y amenazas. La ciberseguridad se ocupa de proteger los sistemas y redes contra ataques cibernéticos que buscan acceder, alterar o destruir información, extorsionar a los usuarios, o interrumpir el funcionamiento de una organización. Esto incluye la seguridad de redes, dispositivos, datos y aplicaciones, así como la gestión de identidades y acceso. Las principales amenazas en ciberseguridad incluyen ataques avanzados persistentes, ransomware, phishing y malware, y se mitigan mediante la implementación de firewalls, sistemas de detección de intrusos, actualizaciones constantes, y la capacitación de los empleados en prácticas de seguridad.\\

Por otro lado, la seguridad informática se centra en la protección de los recursos tecnológicos, como hardware, software e infraestructura, contra accesos no autorizados y daños. Se enfoca en el control de acceso, la gestión de vulnerabilidades, el monitoreo y detección de amenazas, y la respuesta rápida ante incidentes para minimizar el impacto. Los desafíos incluyen la actualización continua de sistemas, la gestión de amenazas internas y la seguridad en una diversidad de dispositivos.\\

La seguridad de la información, por su parte, tiene como objetivo proteger la confidencialidad, integridad y disponibilidad de los datos, independientemente de su formato. Esto implica el desarrollo de políticas de seguridad, la gestión de riesgos, el control de acceso, y la realización de auditorías para asegurar el cumplimiento normativo. Entre los desafíos principales se encuentran la protección de datos en entornos híbridos, el cumplimiento de normativas de privacidad, y la gestión de amenazas tanto internas como externas.\\
\section{Palabras Clave}
\subsection{Objetivo}
El objetivo de conocer y entender los conceptos de ciberseguridad, seguridad informática y seguridad de la información es poder diseñar, implementar y mantener una estrategia integral de protección que salvaguarde tanto los sistemas tecnológicos como los datos sensibles de una organización. Este conocimiento es esencial para anticipar y mitigar riesgos, asegurar la continuidad del negocio, y cumplir con las normativas y regulaciones en un entorno digital cada vez más complejo y amenazado.
\begin{itemize}
    \item Ciberseguridad
    \item Seguridad Informática
    \item Riesgos Cibernéticos
    \item Amenazas Digitales
\end{itemize}

\section{Ciberseguridad}
\onehalfspacing
La ciberseguridad es la práctica de proteger sistemas, redes y programas de ataques digitales. Estos ataques suelen tener como objetivo acceder, alterar o destruir información confidencial, extorsionar a los usuarios o interrumpir la continuidad del negocio.

\subsection{Ámbito de aplicación}
La ciberseguridad abarca todos los aspectos relacionados con la protección de los sistemas digitales frente a amenazas cibernéticas, incluyendo:
\begin{enumerate}
    \item \textbf{Redes:} Protección contra intrusiones, malware, y ataques de denegación de servicio (DDoS).
    \item \textbf{Dispositivos:} Seguridad en computadoras, servidores, móviles y dispositivos IoT (Internet de las Cosas).
    \item \textbf{Datos:} Protección contra accesos no autorizados, cifrado de información y prevención de fuga de datos.
    \item \textbf{Aplicaciones:} Desarrollo seguro, control de acceso y protección de aplicaciones frente a vulnerabilidades.
\end{enumerate}

\subsection{Tipos de ciberseguridad}
\begin{enumerate}
    \item \textbf{Seguridad de red:} Protección de las redes internas y externas, asegurando que solo usuarios y dispositivos autorizados puedan acceder.
    \item \textbf{Seguridad de aplicaciones:} Garantizar que las aplicaciones sean seguras desde el diseño y que las actualizaciones no introduzcan vulnerabilidades.
    \item \textbf{Seguridad en la nube:} Protección de los servicios y datos alojados en la nube, asegurando el cumplimiento de normas y prácticas de seguridad.
    \item \textbf{Seguridad de dispositivos móviles:} Protección contra amenazas específicas a dispositivos móviles, como malware y aplicaciones maliciosas.
    \item \textbf{Gestión de identidad y acceso (IAM):} Controlar quién tiene acceso a qué recursos y garantizar que solo usuarios autorizados puedan acceder a datos sensibles.
\end{enumerate}

\subsection{Desafíos y amenazas}
\begin{enumerate}
    \item \textbf{Amenazas avanzadas persistentes (APT):} Ataques prolongados y dirigidos contra objetivos específicos como gobiernos y grandes corporaciones.
    \item \textbf{Ransomware:} Secuestro de datos mediante cifrado, con exigencia de un rescate para liberarlos.
    \item \textbf{Phishing:} Técnicas de engaño para obtener información confidencial mediante correos electrónicos o sitios web falsos.
    \item \textbf{Malware:} Software malicioso que puede dañar sistemas, robar información o interrumpir operaciones.
\end{enumerate}

\subsection{Mejores prácticas}
\begin{enumerate}
    \item Implementar firewalls y sistemas de detección de intrusos (IDS/IPS).
    \item Mantener sistemas y software actualizados con los últimos parches de seguridad.
    \item Capacitar a los empleados sobre seguridad cibernética y buenas prácticas en el uso de la tecnología.
    \item Implementar autenticación multifactor (MFA) para fortalecer el acceso a sistemas críticos.
\end{enumerate}

\section{Seguridad Informática}
La seguridad informática se centra en la protección de los recursos tecnológicos, como el hardware, el software y la infraestructura, contra accesos no autorizados, ataques, y daños.
\subsection{Ámbito de aplicación}
\begin{enumerate}
    \item \textbf{Hardware:} Protección física y lógica de servidores, computadoras, dispositivos de almacenamiento, y otros equipos.
    \item \textbf{Software:} Garantizar que los programas y sistemas operativos funcionen de manera segura, sin vulnerabilidades que puedan ser explotadas.
    \item \textbf{Infraestructura de red:} Asegurar que routers, switches, cables, y otros componentes de la red estén protegidos contra manipulación y ataques.
\end{enumerate}

\subsection{Aspectos clave}
\begin{itemize}
    \item \textbf{Control de acceso:} Establecer políticas que determinen quién puede acceder a qué recursos y en qué circunstancias.
    \item \textbf{Gestión de vulnerabilidades:} Identificación, evaluación, y mitigación de debilidades en el hardware y software antes de que sean explotadas.
    \item \textbf{Monitoreo y detección:} Supervisar sistemas y redes en busca de actividades sospechosas o inusuales que podrían indicar un intento de ataque.
    \item \textbf{Respuesta a incidentes:} Tener un plan de acción claro para responder rápidamente a ataques o brechas de seguridad, minimizando el impacto.
\end{itemize}

\section{Seguridad de la Información}
La seguridad de la información se refiere a la protección de la confidencialidad, integridad y disponibilidad (los tres pilares de la seguridad de la información) de los datos, independientemente de su formato (digital, impreso, oral, etc.).

\subsection{Ámbito de aplicación}
\begin{itemize}
    \item Confidencialidad: Garantizar que la información esté disponible solo para aquellos que tienen autorización para acceder a ella.
    \item Integridad: Asegurar que la información no sea alterada de manera no autorizada y que se mantenga precisa y completa.
    \item Disponibilidad: Asegurar que los datos estén accesibles y utilizables cuando los usuarios autorizados lo necesiten.
\end{itemize}

\subsection{Componentes clave}
\begin{itemize}
    \item Políticas de seguridad: Establecer directrices sobre cómo se debe proteger la información dentro de una organización.
    \item Gestión de riesgos: Identificar, evaluar y mitigar los riesgos asociados con la protección de la información.
    \item Control de acceso: Establecer y gestionar los permisos para asegurar que solo las personas adecuadas puedan acceder a información sensible.
    \item Auditorías y cumplimiento: Realizar revisiones periódicas para asegurar que las políticas de seguridad se siguen y que se cumplen las normativas legales y reglamentarias.
\end{itemize}

\section{Observaciones y comentarios}
La Ciberseguridad, la Seguridad Informática y la Seguridad de la Información son conceptos interrelacionados que, en conjunto, protegen los sistemas, las redes y los datos en el entorno digital. Mientras que la ciberseguridad se enfoca en proteger contra amenazas cibernéticas, la seguridad informática se centra en la integridad de los recursos tecnológicos, y la seguridad de la información en la protección de los datos en todas sus formas.
\section{Conclusiones:}
\begin{itemize}
    \item Es importante destacar que aunque la ciberseguridad, la seguridad informática y la seguridad de la información se abordan como disciplinas separadas, en la práctica se integran y se solapan en muchos aspectos. 
    \item Las amenazas en el ámbito digital evolucionan constantemente, lo que exige que las organizaciones y los profesionales de la seguridad se mantengan actualizados y adopten tecnologías emergentes y enfoques proactivos para mitigar riesgos.
\end{itemize}

\section{Referecias:}
\begin{itemize}
    \item LISA Institute. (2024). Diferencia entre ciberseguridad, seguridad informática y seguridad de la información. LISA Institute. https://www.lisainstitute.com/blogs/blog/diferencia-ciberseguridad-seguridad-informatica-seguridad-informacion
    \item AO Kaspersky Lab. (2024). ¿Qué es la ciberseguridad? Kaspersky. https://latam.kaspersky.com/resource-center/definitions/what-is-cyber-security
    \item Microsoft. (s.f.). ¿Qué es la ciberseguridad? Microsoft. https://www.microsoft.com/es-es/security/business/security-101/what-is-cybersecurity
\end{itemize}

Link GitHub: https://github.com/PJBigBoss115/Seminario_TIs.git

\end{document}